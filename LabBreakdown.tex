\documentclass[11pt, a4paper]{article}
%\usepackage{geometry}
\usepackage[inner=1.5cm,outer=1.5cm,top=2.5cm,bottom=2.5cm]{geometry}
\pagestyle{empty}
\usepackage{graphicx}
\usepackage{fancyhdr, lastpage, bbding, pmboxdraw}
\usepackage[usenames,dvipsnames]{color}
\definecolor{darkblue}{rgb}{0,0,.6}
\definecolor{darkred}{rgb}{.7,0,0}
\definecolor{darkgreen}{rgb}{0,.6,0}
\definecolor{red}{rgb}{.98,0,0}
\usepackage[colorlinks,pagebackref,pdfusetitle,urlcolor=darkblue,citecolor=darkblue,linkcolor=darkred,bookmarksnumbered,plainpages=false]{hyperref}
\renewcommand{\thefootnote}{\fnsymbol{footnote}}

\pagestyle{fancyplain}
\fancyhf{}
\lhead{ \fancyplain{}{Vertebrate Biodiversity Lab- Lab Breakdown} }
%\chead{ \fancyplain{}{} }
\rhead{ \fancyplain{}{\today} }
%\rfoot{\fancyplain{}{page \thepage\ of \pageref{LastPage}}}
\fancyfoot[RO, LE] {page \thepage\ of \pageref{LastPage} }
\thispagestyle{plain}

%%%%%%%%%%%% LISTING %%%
\usepackage{listings}
\usepackage{caption}
\DeclareCaptionFont{white}{\color{white}}
\DeclareCaptionFormat{listing}{\colorbox{gray}{\parbox{\textwidth}{#1#2#3}}}
\captionsetup[lstlisting]{format=listing,labelfont=white,textfont=white}
\usepackage{verbatim} % used to display code
\usepackage{fancyvrb}
\usepackage{acronym}
\usepackage{amsthm}
\usepackage{arydshln} % dashed line in table
\VerbatimFootnotes % Required, otherwise verbatim does not work in footnotes!



\definecolor{OliveGreen}{cmyk}{0.64,0,0.95,0.40}
\definecolor{CadetBlue}{cmyk}{0.62,0.57,0.23,0}
\definecolor{lightlightgray}{gray}{0.93}



\lstset{
%language=bash,                          % Code langugage
basicstyle=\ttfamily,                   % Code font, Examples: \footnotesize, \ttfamily
keywordstyle=\color{OliveGreen},        % Keywords font ('*' = uppercase)
commentstyle=\color{gray},              % Comments font
numbers=left,                           % Line nums position
numberstyle=\tiny,                      % Line-numbers fonts
stepnumber=1,                           % Step between two line-numbers
numbersep=5pt,                          % How far are line-numbers from code
backgroundcolor=\color{lightlightgray}, % Choose background color
frame=none,                             % A frame around the code
tabsize=2,                              % Default tab size
captionpos=t,                           % Caption-position = bottom
breaklines=true,                        % Automatic line breaking?
breakatwhitespace=false,                % Automatic breaks only at whitespace?
showspaces=false,                       % Dont make spaces visible
showtabs=false,                         % Dont make tabls visible
columns=flexible,                       % Column format
morekeywords={__global__, __device__},  % CUDA specific keywords
}

%%%%%%%%%%%%%%%%%%%%%%%%%%%%%%%%%%%%
\begin{document}

\pagestyle{fancyplain}
\fancyhf{}
\lhead{ \fancyplain{}{BIOL 4020 – Vertebrate Biodiversity - Lab Breakdown}}
%\chead{ \fancyplain{}{}}
\rhead{ \fancyplain{}{Fall 2020}}
%\rfoot{\fancyplain{}{page \thepage\ of \pageref{LastPage}}}
\fancyfoot[RO, LE] {page \thepage\ of \pageref{LastPage}}
\thispagestyle{plain}

\begin{center}
\rule{6in}{0.4pt}
\begin{minipage}[t]{.75\textwidth}
\begin{tabular}{llcccll}
\textbf{GTAs:} & Randy Klabacka: \href{mailto:rlk0015@auburn.edu}{rlk0015@auburn.edu} (702)882-6292 \\ & Morgan Muell: \href{mailto:mrm0161@auburn.edu}{mrm0161@auburn.edu} \\ \textbf{Sections:} & 1: Mon 12:00 -- 2:50 \\ & 2: Tue 12:30 -- 3:15 \\ & 3: Wed 12:00 -- 2:50
\end{tabular}
\end{minipage}
\rule{6in}{0.4pt}
\end{center}
\vspace{.5cm}
\setlength{\unitlength}{1in}
\renewcommand{\arraystretch}{2}

\vskip.15in

\noindent\textbf{Grade Breakdown:}
\begin{table}[h]
\centering
\label{Table:GradeBreakdown}
\begin{tabular}{lll}\hline
Category                     &         \\ \hline
Attendance \& Participation  & 80      \\
Project Proposal             & 40      \\
Field Notebook               & 40      \\
Exams (16 pts each, 5 exams) & 80      \\ \hdashline
Total                        & 240     \\ \hline
\end{tabular}
\bigskip{}
\end{table}

* Due to the COVID-19 pandemic, labs (with the exception of field trips- see \hyperref[SecFieldNote]{Field Notebook Section}) will be conducted remotely using Canvas and zoom. You can complete all of the labs on your own time with the exception of: (1) the lab introduction, (2) your assigned field trip, (3) the proposal workshop, and (4) the proposal panel. To see which field trip you are assigned to, see the document on Canvas under $<$InsertPathToDocumentHere$>$. The lab introduction, proposal workshop, and proposal panel will be conducted over zoom during your scheduled lab time. Details on these items will be provided later. Here is the zoom meeting information:
\begin{itemize}
	\item{zoom meeting id: 954 8133 6875}
	\item{zoom meeting pw: VB4020}
\end{itemize}
For many of the other labs, you will be required to watch a pre-recorded lecture given by one of your TA's and complete the activities specified for that week's lab module (see "Modules" tab on Canvas to see the lab modules for each week).

\vskip.15in
\noindent\textbf{Attendance \& Participation:} %\footnotemark
\begin{itemize}
\item{Participation: \~7 points per lab}
\item{Each lab has a corresponding lab module on Canvas with instructions on how to obtain points for that lab. Obtaining attendance/participation points for each lab is dependent on completion of the elements within each lab module.}
\item{Each lab module will close at the beginning of the following lab (e.g., the Lab Module for Lab 2 will close once the Lab Module for Lab 3 opens). Lab modules open at the start of lab (i.e., 12:00 pm on Monday for Section 1) and end at the beginning of the following lab (e.g., 11:59 am on the following Monday for Section 1)}.
\item{Failure to attend/participate in a lab results in a loss of all points for that lab.}
\end{itemize} 

\vskip.15in
\noindent\textbf{Project Proposal:} %\footnotemark
\begin{itemize}
\item{Rough Draft: 15 points}
\item{Final Draft: 25 points}
\item{Additional details for this assignment are in the \hyperref[SecProjProp]{Project Proposal Section}}
\end{itemize} 

\vskip.15in
\noindent\textbf{Field Notebook:} %\footnotemark
\begin{itemize}
\item{12 hours of in-field observation are required to receive full credit for this assignment}
\item{Hours from lab field weeks contribute to this amount (4 trips: 8 hours)}
\item{4 points will be deducted for each hour a student is short of the 12 total hours}
\item{Additional details for this assignment are in the \hyperref[SecFieldNote]{Field Notebook Section}}
\end{itemize} 

\vskip.15in
\noindent\textbf{Exams:} %\footnotemark
\begin{itemize}
\item{4 exams (20 points each)}
\item{Exams will be administered electronically over Canvas and will include:}
\begin{enumerate}
\item{Photo identification (common names acceptable)}
\item{Natural history facts discussed in lab/field}
\item{Major points from publications discussed in lab}
\item{Phylogenies for focal taxa}
\end{enumerate}
\end{itemize} 


\begin{table}[H]
\begin{singlespace}
\centering
\label{Table:Schedule}
\begin{tabular}{llll}\hline
Week of & Lab Module          & Topic                           & Items due before class  \\ \hline
Aug 17  & Lab 1               & Live zoom overview*             &                         \\
Aug 24  & Lab 2               & Intro to Fishes                 &                         \\
Aug 31  & Lab 3               & Field Trip- Euphapee Creek      & Discussion Template     \\
Sep 7   & \textit{No Lab}     & \textit{Labor Day}              &                         \\
Sep 14  & Lab 4               & Intro to Amphibians             & \textbf{Fish Exam}      \\
Sep 21  & Lab 5               & How To Science                  & Proposal Ideas          \\
Sep 28  & Lab 6               & Field Trip- Opacum Pond         & Canvas Discussion       \\
Oct 5   & Lab 7               & Intro to Diapsids               & \textbf{Amphibian Exam} \\
Oct 12  & Lab 8               & Field Trip- Wood Duck Preserve  & Proposal Rough Draft    \\
Oct 19  & Lab 9               & Proposal Workshop*              & Canvas Discussion       \\
Oct 26  & Lab 10              & Intro to Mammals                & \textbf{Diapsid Exam}   \\
Nov 2   & Lab 11              & Field Trip- Oxbow Pond          & Proposal Submission     \\
Nov 9   & Lab 12              & Proposal Panels*                & Proposal Review         \\
Nov 16  & \textit{No Lab}     & \textit{Study for exam!}        & \textbf{Mammal Exam}    \\ \hline
\end{tabular}
\begin{tablenotes}
\item{* denotes labs with live zoom meetings}
\item{zoom meeting id: 690 238 9071}
\item{zoom meeting pw: VertBio20}
\item{zoom meeting link: \href{https://us02web.zoom.us/j/6902389071?pwd=MXA3eGVCbkw5djBZUDQ2NWU2bGtLdz09}{ZoomLink}}
\end{tablenotes}
\end{singlespace}
\end{table}

\noindent\textbf{Field Behavior and Safety:}
On field trips, we enter the habitats that are home to diverse organisms. It is expected that you will show respect by handling organisms with care, replacing flipped logs to their original position, refrain from littering, and avoid shouting. At all times you must stay with the group as a whole- i.e., you must be able to see other lab members at all times. Given the current pandemic, it is also required that you wear a face mask and maintain 6 ft of physical separation. You are also required to wear close-toed shoes into the field, and it is recommended that you have sun protection and wear clothing that you can get wet/dirty! You will be hiking through dirt, mud, and water. Expect to get knee-deep in water on all field trips. Be cautious and aware as you walk through the woods. Your safety is your responsibility on these trips- if you feel uncomfortable doing something out of concern for your safety, do not do it!\\

\noindent\textbf{Academic Honesty and Inclusion:}
This lab welcomes, respects, and serves students of diverse backgrounds and perspectives, and it is expected that students respect one another. Any acts of aggression or misconduct based on race, color, veteran status, religion, age, rural/urban/national origin, sex or sexual orientation, gender identity, or disability will not be tolerated. Academic dishonesty in any form will also not be tolerated. Given the current status of the global pandemic, you will have more independence than is typically granted in this lab. We ask that you honor the independence granted with your honesty on all lab assignments and exams.


\end{document} 


